\documentclass[10pt]{article}
\usepackage{tikz-cd}
\tikzcdset{labels={font=\everymath\expandafter{\the\everymath\textstyle}}}
\usepackage[a4paper, margin=1in, top=0.7in, columnsep=2cm]{geometry}
\usepackage[yyyymmdd,hhmmss]{datetime}
\usepackage{hyperref}
\usepackage{cool}
\usepackage{fancyhdr}
\usepackage{amsthm}

\usepackage[makeroom]{cancel}

\usepackage{mathpazo}
\usepackage[scaled=0.95]{helvet}
\usepackage{courier}
\linespread{1.05} % Palatino looks better with this


\usepackage{titling}
\setlength{\droptitle}{-5em}   % This is your set screw
\setlength{\parindent}{0ex}
\setlength{\parskip}{0.5em}

\hypersetup{
    colorlinks,
    linkcolor={red!30!black},
    citecolor={blue!50!black},
    urlcolor={blue!80!black}
}

\newtheoremstyle{defstyle}% <name>
{6pt}% <Space above>
{6pt}% <Space below>
{}% <Body font>
{}% <Indent amount>
{}% <Theorem head font>
{:}% <Punctuation after theorem head>
{.5em}% <Space after theorem headi>
{\thmname{\textbf{#1}}\thmnumber{ #2}\thmnote{ (\textsc{#3})}}


\theoremstyle{defstyle}
\newcounter{axiom} \numberwithin{axiom}{section}
\newcounter{examplec} \numberwithin{examplec}{section}
\newtheorem{theorem}{Theorem}[section]
\newtheorem{elemma}[theorem]{Lemma}%[section]
\newtheorem{eclaim}[theorem]{Claim}
\newtheorem{eaxiom}[axiom]{Axiom}
\newtheorem{definition}[theorem]{Definition}
\newtheorem{example}[theorem]{Example}

\newcommand\claim[1]{
  \begin{eclaim}
    #1
  \end{eclaim}
  \bigskip
}

\newcommand\nclaim[2]{
  \begin{eclaim}[#1]
    #2
  \end{eclaim}
  \bigskip
}

\newcommand\defn[2]{
  \begin{definition}[#1]
    #2
  \end{definition}
  \medskip
}

\newcommand\axiom[2]{
  \begin{eaxiom}[#1]
    #2
  \end{eaxiom}
  \medskip
}

\newcommand\lemma[3]{
  \begin{elemma}[#1]
    #2
  \end{elemma}
  \begin{proof}
    #3
  \end{proof}
  \medskip
}

\pagestyle{fancy}
\lhead{}
\chead{}
\rhead{}
\lfoot{\footnotesize {\yyyymmdddate\today} at \currenttime}
\cfoot{}
\rfoot{\thepage}
\renewcommand{\headrulewidth}{0.0pt}
\renewcommand{\footrulewidth}{0.0pt}

\usepackage{graphicx}
\usepackage{float} % use [H] for placing
\graphicspath{ {figures/} }

\usepackage{amsmath,amsthm,amssymb}
\usepackage{mathtools}
\usepackage{xcolor}
\usepackage{hyperref}
\usepackage{xspace}
\usepackage{comment}
\usepackage{url} % for bib entries
\usepackage{fancyhdr}

%% no paragraph offsets
\usepackage{parskip}
\setlength{\parskip}{0.1cm}
\setlength{\parindent}{0mm}

\newcommand{\rd}[1]{\ref{def:#1}}

\newcommand{\todo}[1]{\textcolor{gray}{\bf{#1}}} % TODO note
\newcommand{\attr}[1]{\textcolor{gray}{\bf{#1}}} % attribution

% bold vector
\renewcommand{\vec}[1]{\mathbf{#1}}

% common definitions

\newcommand*{\op}[1]{\operatorname{#1}}
\newcommand*{\vr}[1]{\mathbb{R}^{#1}}
\newcommand\N{\mathcal{N}}
\newcommand\inv[1]{{#1}^{-1}}
\newcommand\abs[1]{\left|#1\right|}
\newcommand\e[1]{\mathrm{e}^{#1}}
\renewcommand\exp[1]{\mathbf{exp}\,\left\{#1\right\}}
\newcommand\like{\propto}


\title{Introduction to Homotopy Theory \\
  \large (Notes based on lectures by Sergiy Maksymenko)
}
\author{}
\date{}


\begin{document}

\maketitle
\thispagestyle{fancy}

\begin{abstract}
  Homotopy theory studies spaces up to a homotopy, which is a continuous deformation of one
  continuous function to another. This documents is a work in progress done during a course audit.
  These notes are taken purposefully in English to strenghten intuition
  and simplify lookup of concepts in related literature.

  Warning: this document may be edited live during audit so watch out for incorrect statements!
\end{abstract}

\tableofcontents

\section{Set-theoretic Definitions}

\defn{Binary relation}{
  A binary relation R on a set $X$ is a set of ordered pairs of elements of $X$.
}

\defn{Equivalence relation}{\label{def:equivalence}
  An equivalence relation $\sim$ is a binary relation that is reflexive, symmetric and transitive.
}

\let\OldSim\sim
\renewcommand{\sim}{\hyperref[def:equivalence]{\OldSim}}

\defn{Equivalence class of an element}{\label{def:eqclass}
  Given a set $X$ and an equivalence relation $\sim$, an
  equivalence class of $a \in X$, denoted $[a]$ is a set $\{ x \in S \mid x \sim a \}$
}

\newcommand{\eqclass}[1]{\hyperref[def:eqclass]{[#1]}}

\defn{Quotient Set}{\label{def:quotient-set}
  A quotient set $X/{\sim}$ (also said ``$X$ \textit{modulo} $\sim$'') is a set of all \hyperref[def:eqclass]{equivalence
    classes} of $X$ with respect to $\sim$.
  $$
  X/{\sim} = \{ \eqclass{x} : x \in X \}
  $$
}

\newcommand{\quot}[2]{\hyperref[def:quotient-set]{#1/{#2}}}

\defn{Quotient Map}{\label{def:quotient-map}
  A quotient map is a surjective mapping that sends a point in $X$ to its equivalence class, containing it:
  $q: X \to \quot{X}{\sim}$
}

\defn{Quotient by Set Membership}{\todo{TODO}. $\quot{X}{A}$}


\section{Point-set Topology}

\newcommand{\interval}{\ensuremath{[0..1]}}

\defn{Topological space}{\label{def:topology}
  A topological space is a pair $\langle X, \tau \rangle$, where $X$ is a set
  and $\tau$, a topology on X, is a collection of subsets ($\tau \subseteq \mathcal{P}(X)$)
  called open sets, such that:

  \begin{itemize}
  \item $\emptyset \in \tau$.
  \item $X \in \tau$.
  \item $\tau$ is closed under arbitrary finite intersections.
  \item $\tau$ is closed under arbitrary unions. \todo{Maybe find cute notation for this.}
  \end{itemize}
}

\todo{Posets}.

\newcommand{\topX}{\hyperref[def:topology]{X}}
\renewcommand{\top}[2]{\hyperref[def:topology]{\pair{#1}{#2}}}

\defn{Trivial Topology}{
  A topological space is called trivial, when the topology on $X$
  consists only of $\emptyset$ and $X$.
}

\defn{Discrete Topology}{
  A topological space is called discrete, when $\tau = \mathcal{P}(X)$.
}

\defn{Continuous map}{
  Let $\langle X, \tau \rangle$ and $\langle Y, \sigma \rangle$ be topological spaces.
  A map $f: X \to Y$ is \textbf{continuous} if:
  $$
  \forall s \in \sigma, \inv{f}(s) \in \tau
  $$
  In plain English, a map is continuous when a preimage of an open set in $Y$ is an open set in $X$.

  $C(X,Y)$ denotes a set of all continuous maps between $X$ and $Y$.
}

\todo{Base of topology and methods of inducing topologies on sets were discussed during Lecture 2.}

\todo{Bonus. Compare topology to a field of sets to a $\sigma$-algebra to a Borel $\sigma$-algebra. Discussed during Lecture 5.}

\subsection{Topology Restrictions}

\defn{$T_1$ space}{}
\defn{Hausdorff space}{}

\subsection{Quotient Topology}

\todo{Gluing.}

Let $\top{X}{\tau}$ be a topological space and $\sim$ be an equivalence relation on $X$.

\defn{Quotient Topogical Space}{
  By analogy of a set and given $q: \top{X}{\tau} \to \top{\quot{X}{\sim}}{\tau_{\quot{X}{\sim}}}$,
  \begin{align*}
    \quot{X}{\sim} & = \{ \eqclass{x} : x \in X \} \\
    \tau_{\quot{X}{\sim}} & = \{ U \subseteq \quot{X}{\sim} \mid \inv{q}(U) \in \tau \}
  \end{align*}

  $\tau_{\quot{X}{\sim}}$ is constructed this way to ensure $q$ is continuous.
}

\begin{theorem}
  Given $Z$ is a topological space, $g$ is a surjective map, $p$ is a quotient map and the following diagram,

  % \[
  %   \begin{tikzcd}[font=\large, row sep=huge, column sep=huge]
  %     \topX \arrow[swap]{d}{p} \arrow{dr}{g \circ p} &
  %     \\
  %     \quot{X}{\sim} \arrow[swap]{r}{g} & Z
  %   \end{tikzcd}
  % \]

  \[
    \begin{tikzcd}[font=\large, row sep=huge, column sep=huge]
      \topX \arrow{dr}{p} \arrow{rr}{f = g \circ p} && Z \\
      & \quot{X}{\sim} \arrow{ur}{g}
    \end{tikzcd}
  \]

  $g$ is continuous $\iff$ $f = g \circ p$ is continuous.
\end{theorem}

\begin{example}
  Have $z \in \mathbb{C}$, $\mathbb{S}^1 = \{ \abs{z} = 1 \}$, $I = [0..1] \subset \vr{1}$.

  \[
    \begin{tikzcd}[font=\large, row sep=huge, column sep=huge]
      I \arrow{dr}{p} \arrow{rr}{f(t\in I) = \e{2i\pi{}t}} && \mathbb{S}^1 \\
      & {\quot{I}{\Delta} = \{0,1\}} \arrow{ur}{g}
    \end{tikzcd}
  \]
\end{example}



\section{Homotopies Between Continuous Maps}

\defn{Homotopy}{
  Two continuous maps $f, g: X \to Y$ are homotopic if there is a
  map called homotopy $H: X \times [0,1] \to Y$ that \textit{continuously deforms} $f$ to $g$,
  denoted $f \simeq g$ or $f \homotopic{H} g$.
  In general:
  \begin{align*}
    X \times [0,1] &\xrightarrow{H} Y \\
    H(x, 0) &= f(x) \\
    H(x, 1) &= g(x) \\
    H(x, t) &= tf(x) + (1-t)g(x)
  \end{align*}
}

\begin{example}
$1 \homotopic{x^t} x$ when viewed as $x^0$ and $x^1$.
\begin{align*}
  X \times [0,1] &\xrightarrow{H} Y \\
  H(x, 0) &= x^0 \\
  H(x, 1) &= x^1 \\
  H(x, t) &= x^t
\end{align*}
Another possible homotopy between the same functions is $t\cdot{}x + (1-t)\cdot{}1$, which suggsts that there may be many more of them.

\todo{Plots?}
\end{example}

\begin{example}
  $\{\cdot\} \times [0,1] \to \mathbb{C}$ with $H(x,t) = e^{2\pi{}it}$
\end{example}

\begin{theorem}
  A homotopy between continuous maps is an \nameref{def:equivalence}.
\end{theorem}

\begin{proof}
  \ldots
\end{proof}

\subsection{Contractible Spaces}

\defn{Contractibility}{
  A space is contractible if it is homotopically equivalent to a point (a constant map).

  Extra definitions:
  \begin{itemize}
  \item \todo{$X\times 0$ is a retraction of CX (cone over X)}
  \item \todo{homotopic equivalence to a point}
  \end{itemize}

}

\begin{example}
  $\vr{n}$ is contractible to a point.
  \begin{align*}
  \vr{n} \times [0,1] &\xrightarrow{H} \vr{n} \\
  H(x, 0) &= 0 \\
  H(x, 1) &= x \\
  H(x, t) &= tx
  \end{align*}
\end{example}


\defn{Path-connectedness}{
}


\begin{theorem}
  Any convex set is contractible.
\end{theorem}


\begin{elemma}
  Contractibility does not depend on a choice of a point.
\end{elemma}

\defn{Star-convex Set}{}

\begin{theorem}
  A star-convex set is contractible to a point.
  \begin{align*}
    A \subset \vr{n} &\text{\;is (star-)convex} \\
    a \in A \\
    A \times [0,1] &\xrightarrow{H} A \\
    H(x, 0) &= 0 \\
    H(x, 1) &= x \\
    H(x, t) &= at + x(1-t)
  \end{align*}
\end{theorem}

\begin{theorem}
  Assuming we know how to build topologies on trees (as in graph theory trees), every finite tree is a contractible topological space.
\end{theorem}



\section{Quotient Spaces and Maps}

\todo{See lectures 2 and 3. Most of this stuff}

\subsection{Quotients and Groups}

\subsection{Cones of Topological Spaces}

\todo{Brouwer's Fixed Point Theorem was mentioned around here. Bring up the context?}.

\section{Retractions, Deformations and Deformation Retractions}

\defn{Retraction}{}

\todo{Bring up the example with ${x \over \norm{x}} - discussed along with retractions$}

\defn{Deformation}{
  A continuous mapping is a deformation into a subspace $A \subset X$ when
  \begin{itemize}
  \item $H_0 = id_X$
  \item $\forall t \in \interval, H_t(A) \subset A$
  \end{itemize}
}

\defn{Deformation Retraction}{}

\section{Classes of Homotopy Maps}

\todo{See lectures 4 and 5}.

\subsection{Mappings of $\mathbb{S}^1$ to Itself}


\section{Homotopy Types}

Subtitle: Homotopy Equivalences Between Spaces. Note that omotopy types have nothing to do with HOTT.


Let $\top{X}{\tau}$ $\top{Y}{\sigma}$

- homeomorphisms and commutative diagrams
- idea: homeomorphisms and topological spaces form a category (TODO illustration here), and a group (for all morphisms with one side fixed)


$Homeo(X)$ group of all identity maps of X

\texttt{g} is a retraction of $Y$ to $X$ on $g(x)$ , g is also left inverse of f

$ g \circ f \homotopic{} id_x $

\defn{Homotopic Equivalence}{

  A mapping $f: X \to Y$ is a homotopic equivalence when $\exists g: Y \to X$ such that $ g \circ f \homotopic{} id_X $ and $ f \circ g \homotopic{} id_Y $
}.


\begin{theorem}
  Homotopy equivalence between spaces is an \nameref{def:equivalence}. \attr{Discussed during lecture 7}.
\end{theorem}

\begin{proof}
  \ldots
\end{proof}

\let\point\cdot

\begin{example}
  $f: X \to \point $ is a homotopic equivalence to a point $\iff$ X is contractible.

  $H_0 = id \homotopic{} H_1 = f$

\end{example}

  % \[
  %   \begin{tikzcd}[font=\large, row sep=huge, column sep=huge]
  %      X \arrow{r}{$f$} Y \arrow{r}{$g$} X
  %   \end{tikzcd}
  % \]

\claim{Holes matter! $\infty \homotopic{} \texttt{B} \homotopic{} \textit{o\_O}$}


\begin{example}
  Glueing of a disk to some space using quotienting.
  \textit{expanding/reducing?}
\end{example}

\claim{Homotopy Theory cares only about spaces glued from a finite number of $\mathcal{S}^n$ spaces, see Shape Theory.}


\defn{Simple Homotopy Type}{A homotopy type is simple if you use a finite number of expansions and reductions}


%foreshadowing
\claim{This is a cue to working with simplicial complexes.}

\subsection*{\ldots Interlude \ldots}

$f: X\to Y$, $f$ is a proper mapping, $\inv{f}(\text{compact})$ is a compact

\todo{Define compact}

Defining proper homotopy equivalences.

$\vr{n} \to \point$ is an improper homotopy. Compactness is a bitch.

\section{Resources}

Course page: \url{https://sites.google.com/site/kafedramatematikikau/products-services/homotopy-theory}

\bibliography{references}{}
\bibliographystyle{apalike}


\end{document}
