\documentclass[10pt]{article}
% \usepackage{parskip}
\usepackage{tikz-cd}
\usepackage{amsmath}
%\usepackage{tocloft}
\tikzcdset{labels={font=\everymath\expandafter{\the\everymath\textstyle}}}
\usepackage[a4paper, margin=1in, top=0.7in, columnsep=2cm]{geometry}
\usepackage[yyyymmdd,hhmmss]{datetime}
\usepackage{hyperref}
\usepackage{cool}
\usepackage{fancyhdr}
\usepackage{amsthm}

\usepackage[makeroom]{cancel}

\usepackage{mathpazo}
\usepackage[scaled=0.95]{helvet}
\usepackage{courier}
\linespread{1.05} % Palatino looks better with this


\usepackage{titling}
\setlength{\droptitle}{-5em}   % This is your set screw
\setlength{\parindent}{0ex}
\setlength{\parskip}{0.5em}

\hypersetup{
    colorlinks,
    linkcolor={red!30!black},
    citecolor={blue!50!black},
    urlcolor={blue!80!black}
}

\newtheoremstyle{defstyle}% <name>
{6pt}% <Space above>
{6pt}% <Space below>
{}% <Body font>
{}% <Indent amount>
{}% <Theorem head font>
{:}% <Punctuation after theorem head>
{.5em}% <Space after theorem headi>
{\thmname{\textbf{#1}}\thmnumber{ #2}\thmnote{ (\textsc{#3})}}


\theoremstyle{defstyle}
\newcounter{axiom} \numberwithin{axiom}{section}
\newcounter{examplec} \numberwithin{examplec}{section}
\newtheorem{theorem}{Theorem}[section]
\newtheorem{elemma}[theorem]{Lemma}%[section]
\newtheorem{eclaim}[theorem]{Claim}
\newtheorem{eaxiom}[axiom]{Axiom}
\newtheorem{definition}[theorem]{Definition}
\newtheorem{example}[theorem]{Example}

\newcommand\claim[1]{
  \begin{eclaim}
    #1
  \end{eclaim}
  \bigskip
}

\newcommand\nclaim[2]{
  \begin{eclaim}[#1]
    #2
  \end{eclaim}
  \bigskip
}

\newcommand\defn[2]{
  \begin{definition}[#1]
    #2
  \end{definition}
  \medskip
}

\newcommand\axiom[2]{
  \begin{eaxiom}[#1]
    #2
  \end{eaxiom}
  \medskip
}

\newcommand\lemma[3]{
  \begin{elemma}[#1]
    #2
  \end{elemma}
  \begin{proof}
    #3
  \end{proof}
  \medskip
}

\pagestyle{fancy}
\lhead{}
\chead{}
\rhead{}
\lfoot{\footnotesize {\yyyymmdddate\today} at \currenttime}
\cfoot{}
\rfoot{\thepage}
\renewcommand{\headrulewidth}{0.0pt}
\renewcommand{\footrulewidth}{0.0pt}

\usepackage{graphicx}
\usepackage{float} % use [H] for placing
\graphicspath{ {figures/} }

\usepackage{amsmath,amsthm,amssymb}
\usepackage{mathtools}
\usepackage{xcolor}
\usepackage{hyperref}
\usepackage{xspace}
\usepackage{comment}
\usepackage{url} % for bib entries
\usepackage{fancyhdr}

%% no paragraph offsets
\usepackage{parskip}
\setlength{\parskip}{0.1cm}
\setlength{\parindent}{0mm}

\newcommand{\rd}[1]{\ref{def:#1}}

\newcommand{\todo}[1]{\textcolor{gray}{\bf{#1}}} % TODO note
\newcommand{\attr}[1]{\textcolor{gray}{\bf{#1}}} % attribution

% bold vector
\renewcommand{\vec}[1]{\mathbf{#1}}

% common definitions

\newcommand*{\op}[1]{\operatorname{#1}}
\newcommand*{\vr}[1]{\mathbb{R}^{#1}}
\newcommand\N{\mathcal{N}}
\newcommand\inv[1]{{#1}^{-1}}
\newcommand\abs[1]{\left|#1\right|}
\newcommand\e[1]{\mathrm{e}^{#1}}
\renewcommand\exp[1]{\mathbf{exp}\,\left\{#1\right\}}
\newcommand\like{\propto}

\newcommand*{\truth}{\top}
\newcommand*{\conj}{\wedge}
\newcommand*{\disj}{\vee}
\newcommand*{\falsehood}{\bot}
\newcommand*{\imp}{\supset}
\newcommand*{\zero}{0}


\title{Introduction to Homotopy Theory \\
  \large (Notes based on lectures by Sergiy Maksymenko)
}
\author{}
\date{}


%\newcommand{\listofdefinitionsname}{List of definitions}
%\newlistof[section]{defns}{dfn}{\listofdefinitionsname}

\begin{document}

\maketitle
\thispagestyle{fancy}

\begin{abstract}
  Homotopy theory studies spaces up to a homotopy, which is a continuous deformation of one
  continuous function to another. This documents is a work in progress done during a course audit.
  These notes are taken purposefully in English to strenghten intuition
  and simplify lookup of concepts in related literature.

  Warning: this document may be edited live during audit so watch out for incorrect statements!
\end{abstract}

\tableofcontents


\newcommand{\integer}[0]{\mathbb{Z}}

\newrobustcmd{\builder}[2]{\{ #1 \bigm| #2 \}}
\newrobustcmd{\angled}[1]{\langle #1 \rangle}

% forward declarations

\newcommand{\wedgesum}{\hyperref[def:wedgesum]{\vee}}
\newcommand{\glue}{\hyperref[def:glueing]{\bigcup}}
\newcommand{\cwcomplex}{\nameref{def:cw-complex}}


\section{Set-theoretic Definitions}

\defn{Binary relation}{
  A binary relation R on a set $X$ is a set of ordered pairs of elements of $X$.
}

\let\OldSim\sim
\renewcommand{\sim}{\hyperref[def:equivalence]{\OldSim}}

\defn{Equivalence relation}{\label{def:equivalence}
  An equivalence relation $\sim$ is a binary relation that is reflexive, symmetric and transitive.
}

\defn{Equivalence class of an element}{\label{def:eqclass}
  Given a set $X$ and an equivalence relation $\sim$, an
  equivalence class of $a \in X$, denoted $[a]$ is a set $\{ x \in S \mid x \sim a \}$
}

\newcommand{\eqclass}[1]{\hyperref[def:eqclass]{[#1]}}

\newrobustcmd{\upto}[1]{\hyperref[def:quotient-set]{#1}}

\defn{Quotient Set}{\label{def:quotient-set}
  A quotient set $X/{\sim}$ (also said ``$X$ \textit{modulo} $\sim$'' or ``$X$ up to $\sim$``) is a set of all \hyperref[def:eqclass]{equivalence
    classes} of $X$ with respect to $\sim$.
  $$
  X/{\sim} = \{ \eqclass{x} : x \in X \}
  $$
}

\newcommand{\quot}[2]{\hyperref[def:quotient-set]{#1\diagup{}{#2}}}

\defn{Quotient Map, Projection}{\label{def:quotient-map}
  A surjective mapping that sends a point in $X$ to its equivalence class, containing it.

  $$\pi: X \to \quot{X}{\sim}$$
}

\newcommand{\quotmemb}[2]{\hyperref[def:quotient-set]{#1\diagup{}{#2}}}

\defn{Quotient by Subset Membership}{\label{def:quotmemb}
  Given a set $X$ and subset $A \subset X$, $\quot{X}{A}$ is a notation that assumes an implicit equivalence relation:

  \[
    \forall \alpha \beta \in X,\; \alpha \sim \beta =
    \begin{cases}
      \top, \alpha \in A \conj \beta \in A \\
      \bot, \text{otherwise}
    \end{cases}
  \]

  For example:
  \begin{align*}
    X & = \{a, b, c, d, e\} \\
    A & = \{a, e\} \subset X \\
    \quot{X}{A} & = \{ b, c, d, \eqclass{a} \} \\
    \eqclass{a} & = \{a, e\}
  \end{align*}

  One can say {\it identification of all points in $A$ with each other}.
}

\section{Group-theoretic Definitions}

\defn{Presentation}{
  $$ \langle S\mid R\rangle$$
  Where $S$ is a generator and $R$ is a relation.
}

\defn{Commutator}{
  That one for groups: \url{https://en.wikipedia.org/wiki/Commutator}
}

\defn{Kernel (group)}{
  Analog of a null space of a linear map (turns out it's also often called a kernel in linear algebra)
}

\begin{example}
  $$ \mathbb{Z}_5 = {0,1,2,3,4} (\mod 5) $$
  %<
  $$ \{e,a,a^2,a^3,a^4\} = \langle a \mid a^5 = e \rangle $$
\end{example}

\defn{Free Group}{
  Syntax for groups! $F_2$ if the free group has two generators.
}

\defn{Free Group by Universal Property}{
  See \url{https://en.wikipedia.org/wiki/Free_group\#Universal_property}.
}

\begin{example}  $F_1 = \integer $ (free cyclic group)
\end{example}

\begin{theorem}
  Let group G be generated $g_1 \ldots g_k$, so every $h\in G$ is a product of some of its generators.

  $$ F_k \xrightarrow{\phi} G $$

  $$ G \cong F_k \diagup \ker \phi $$

  $\ker \phi$ is a list of things on the right.
\end{theorem}

\begin{example}
  {Dihedral group $\mathcal{D}_5$ --- symmetries of the pentagon.}

  $r$ {\it rotates} the pentagon, $s$ {\it flips} it along the vertical axis.

  $\mathcal{D}_5 = \angled{r, s \mid r^5 = 1, s^2 = 1, rs = sr^{-1}} $

  See also: \url{https://en.wikipedia.org/wiki/Dihedral_group#Other_definitions}
\end{example}

\defn{Free Product with Amalgamation}{\label{def:amalgamatedfp}
}

\section{Geometry Definitions}

Most of these constructions are the same \upto{up to} homotopy.

\newrobustcmd{\ndisk}[1]{\hyperref[def:ndisk]{\mathcal{D}^{#1}}}
\newrobustcmd{\nrball}[2]{\hyperref[def:ndisk]{\mathcal{B}_{#1}^{#2}}}
\newrobustcmd{\nsphere}[1]{\hyperref[def:nsphere]{\mathcal{S}^{#1}}}
\newrobustcmd{\ntorus}[1]{\hyperref[def:ntorus]{\mathcal{T}^{#1}}}


\defn{n-sphere, hypersphere}{\label{def:nsphere}
  $\nsphere{n}$. Generalizes a unit circle ($\nsphere{1}$ --- a circle in $\vr{2}$) and a unit sphere ($\nsphere{2}$ --- a sphere in $\vr{3}$). Given $\|\cdot\|$ is a norm,

  $$\nsphere{n} = \builder{x}{\|x \in \vr{n+1}\| = 1} $$

  A unit 0-sphere is a pair of endpoints of an interval $[-1, 1]$ of the real line.
}

\defn{n-disk, n-ball}{\label{def:ndisk}\label{def:ball}
  $\ndisk{n}$. A disk (a closed ball) is a region contained inside $\nsphere{n}$. $\ndisk{1}$ is a line segment (in $\vr{1}$).

  $$\ndisk{n} = \builder{x}{\|x \in \vr{n}\| \leq 1} $$

  A {\bf ball} usually refers to a disk minus its boundary, an {\bf open ball}, which can be defined using $<$ instead of $\leq$.

  $$\nrball{r}{n} = \builder{x}{\|x \in \vr{n}\| < r} $$
}

\defn{sphere modulo disk}{\label{def:spheremod}
  $$
  \nsphere{n} = \quotmemb{\ndisk{n}}{\nsphere{n-1}} = \quotmemb{\ndisk{n}}{\partial\ndisk{n}}
  $$

  For example, identifying the ends (a boundary, which in this case is a zero-dimensional sphere) of a line segment ($\ndisk{1}$) as one point produces a circle ($\nsphere{1}$) --- an object embeddable in $\vr{2}$:

  $$ \nsphere{1} = \quotmemb{\ndisk{1}}{\nsphere{0}} $$

  For a 2-sphere, imagine making a dumpling from a disk-shaped piece of dough. It's not exactly a sphere, but topologically we don't care.

  \[
    \nsphere{2} = \quotmemb{\ndisk{2}}{\nsphere{1}}
  \]
}

\defn{torus}{\label{def:ntorus}
  $\ntorus{n}$.

  A 2-torus is a product of two circles. You can also build it my glueing two opposite sides of a square. Remember that a square is homeomophic to a $\ndisk{2}$. Let $B$ be a thickened bouquet $\nsphere{1} \wedgesum \nsphere{1}$ (so $\nsphere{1} \wedgesum \nsphere{1}$ is a \nameref{def:deformation-retraction} of $B$).

  $$\ntorus{2} = \nsphere{1} \times \nsphere{1} = \ndisk{2} \glue_{f:\; \partial\ndisk{2} \;\to\; \nsphere{1} \wedgesum \nsphere{1}} B $$
}

\defn{double torus}{\label{def:double-torus}
  A donut with two holes. Also known as a \href{https://en.wikipedia.org/wiki/Genus_(mathematics)}{genus}-two surface.
  Can be constructed by folding a hexagon.
}


\section{Point-set Topology}

\newcommand{\interval}{\ensuremath{[0..1]}}

\defn{Unit interval}{A unit interval (or just an interval)  $I$ is a subset $\interval \subset \vr{}$ of a real line, which is a popular notation in further homotopy buildup.}

\defn{A graph of a function}{$\Gamma(f)$ - a subset of $\vr{} \times \vr{}$}


\defn{Topological space}{\label{def:topology}
  A topological space is a pair $\langle X, \tau \rangle$, where $X$ is a set
  and $\tau$, a topology on X, is a collection of subsets ($\tau \subseteq \mathcal{P}(X)$)
  called open sets, such that:

  \begin{itemize}
  \item $\emptyset \in \tau$.
  \item $X \in \tau$.
  \item $\tau$ is closed under arbitrary finite intersections.
  \item $\tau$ is closed under arbitrary unions. \todo{Maybe find cute notation for this.}
  \end{itemize}
}

\todo{Posets}.

\newcommand{\topX}{\hyperref[def:topology]{X}}
\renewcommand{\top}[2]{\hyperref[def:topology]{\pair{#1}{#2}}}

\defn{Trivial Topology}{
  A topological space is called trivial, when the topology on $X$
  consists only of $\emptyset$ and $X$.
}

\defn{Discrete Topology}{
  A topological space is called discrete, when $\tau = \mathcal{P}(X)$.
}

\defn{Continuous map}{
  Let $\langle X, \tau \rangle$ and $\langle Y, \sigma \rangle$ be topological spaces.
  A map $f: X \to Y$ is \textbf{continuous} if:
  $$
  \forall s \in \sigma, \inv{f}(s) \in \tau
  $$
  In plain English, a map is continuous when a preimage of an open set in $Y$ is an open set in $X$.

  $C(X,Y)$ denotes a set of all continuous maps between $X$ and $Y$.
}

\defn{Homeomorphism}{
  A mapping $f: X \to Y$ is a homeomorphism if $\exists g: Y \to X$ s.t. $X \xrightarrow{f} Y \xrightarrow{g} X \xrightarrow{f} Y$,
  when $g \circ f = id_x$ and $f \circ g = id_Y$. It is a continuous map that has a continuous inverse.
}

\begin{example}
  \textit{How to see that a map $h: \interval \to \interval$ is a homeomorphism?}
  \begin{itemize}
  \item At least it needs to be monotonous.
  \item Preserves boundaries.
  \end{itemize}

  $H((\interval, 0, 1), (\interval, 0, 1))_\text{increasing} \bigcup H((\interval, 0, 1), (\interval, 1, 0))_\text{decreasing}$


\end{example}

\todo{Base of topology and methods of inducing topologies on sets were discussed during Lecture 2.}

\todo{Bonus. Compare topology to a field of sets to a $\sigma$-algebra to a Borel $\sigma$-algebra. Discussed during Lecture 5.}

\subsection{Topology Restrictions}

\defn{$T_1$ space}{}
\defn{Hausdorff space}{}


\subsection{Quotient Topology}

\todo{Gluing.}

Let $\top{X}{\tau}$ be a topological space and $\sim$ be an equivalence relation on $X$.

\defn{Quotient Topogical Space}{
  By analogy of a set and given $q: \top{X}{\tau} \to \top{\quot{X}{\sim}}{\tau_{\quot{X}{\sim}}}$,
  \begin{align*}
    \quot{X}{\sim} & = \{ \eqclass{x} : x \in X \} \\
    \tau_{\quot{X}{\sim}} & = \{ U \subseteq \quot{X}{\sim} \mid \inv{q}(U) \in \tau \}
  \end{align*}

  $\tau_{\quot{X}{\sim}}$ is constructed this way to ensure $q$ is continuous.
}

\begin{theorem}
  Given $Z$ is a topological space, $g$ is a surjective map, $p$ is a quotient map and the following diagram,

  % \[
  %   \begin{tikzcd}[font=\large, row sep=huge, column sep=huge]
  %     \topX \arrow[swap]{d}{p} \arrow{dr}{g \circ p} &
  %     \\
  %     \quot{X}{\sim} \arrow[swap]{r}{g} & Z
  %   \end{tikzcd}
  % \]

  \[
    \begin{tikzcd}[font=\large, row sep=huge, column sep=huge]
      \topX \arrow{dr}{p} \arrow{rr}{f = g \circ p} && Z \\
      & \quot{X}{\sim} \arrow{ur}{g}
    \end{tikzcd}
  \]

  $g$ is continuous $\iff$ $f = g \circ p$ is continuous.
\end{theorem}

\defn{Bouqet of spaces, wedge sum}{\label{def:wedgesum}
  Glue two spaces by one point. $A \vee B$.
}

% BAD
% \begin{example}
%   Have $z \in \mathbb{C}$, $\mathbb{S}^1 = \{ \abs{z} = 1 \}$, $I = [0..1] \subset \vr{1}$.

%   \[
%     \begin{tikzcd}[font=\large, row sep=huge, column sep=huge]
%       I \arrow{dr}{p} \arrow{rr}{f(t\in I) = \e{2i\pi{}t}} && \mathbb{S}^1 \\
%       & {\quot{I}{\Delta} = \{0,1\}} \arrow{ur}{g}
%     \end{tikzcd}
%   \]
% \end{example}



\section{Homotopies Between Continuous Maps}

\defn{Homotopy}{
  Two continuous maps $f, g: X \to Y$ are homotopic if there is a
  map called homotopy $H: X \times [0,1] \to Y$ that \textit{continuously deforms} $f$ to $g$,
  denoted $f \simeq g$ or $f \homotopic{H} g$.
  In general:
  \begin{align*}
    X \times [0,1] &\xrightarrow{H} Y \\
    H(x, 0) &= f(x) \\
    H(x, 1) &= g(x) \\
    H(x, t) &= tf(x) + (1-t)g(x)
  \end{align*}
}

\begin{example}
$1 \homotopic{x^t} x$ when viewed as $x^0$ and $x^1$.
\begin{align*}
  X \times [0,1] &\xrightarrow{H} Y \\
  H(x, 0) &= x^0 \\
  H(x, 1) &= x^1 \\
  H(x, t) &= x^t
\end{align*}
Another possible homotopy between the same functions is $t\cdot{}x + (1-t)\cdot{}1$, which suggsts that there may be many more of them.

\todo{Plots?}
\end{example}

\begin{example}
  $\{\cdot\} \times [0,1] \to \mathbb{C}$ with $H(x,t) = e^{2\pi{}it}$
\end{example}

\begin{theorem}
  A homotopy between continuous maps is an \nameref{def:equivalence}.
\end{theorem}

\begin{proof}
  \ldots
\end{proof}

\subsection{Contractible Spaces}

\defn{Contractibility}{
  A space is contractible if it is homotopically equivalent to a point (a constant map).

  Extra definitions:
  \begin{itemize}
  \item \todo{$X\times 0$ is a retraction of CX (cone over X)}
  \item \todo{homotopic equivalence to a point}
  \end{itemize}

}

\begin{example}
  $\vr{n}$ is contractible to a point.
  \begin{align*}
  \vr{n} \times [0,1] &\xrightarrow{H} \vr{n} \\
  H(x, 0) &= 0 \\
  H(x, 1) &= x \\
  H(x, t) &= tx
  \end{align*}
\end{example}


\defn{Path-connectedness}{

}


\begin{theorem}
  Any convex set is contractible.
\end{theorem}


\begin{elemma}
  Contractibility does not depend on a choice of a point.
\end{elemma}

\defn{Star-convex Set}{}

\begin{theorem}
  A star-convex set is contractible to a point.
  \begin{align*}
    A \subset \vr{n} &\text{\;is (star-)convex} \\
    a \in A \\
    A \times [0,1] &\xrightarrow{H} A \\
    H(x, 0) &= 0 \\
    H(x, 1) &= x \\
    H(x, t) &= at + x(1-t)
  \end{align*}
\end{theorem}

\begin{theorem}
  Assuming we know how to build topologies on trees (as in graph theory trees), every finite tree is a contractible topological space.
\end{theorem}



\section{Quotient Spaces and Maps}

\todo{See lectures 2 and 3. Most of this stuff}

\subsection{Quotients and Groups}

\subsection{Cones of Topological Spaces}

\todo{Brouwer's Fixed Point Theorem was mentioned around here. Bring up the context?}.

\section{Retractions, Deformations and Deformation Retractions}

\defn{Retraction}{}

\todo{Bring up the example with ${x \over \norm{x}} - discussed along with retractions$}

\defn{Deformation}{
  A continuous mapping is a deformation into a subspace $A \subset X$ when
  \begin{itemize}
  \item $H_0 = id_X$
  \item $\forall t \in \interval, H_t(A) \subset A$
  \end{itemize}
}

\defn{Deformation Retraction}{\label{def:deformation-retraction}}

\section{Classes of Homotopy Maps}

\todo{See lectures 4 and 5}.

% {
% %  \medskip
%   \includegraphics[width=0.4\textwidth]{Z}
% %  \medskip
% }

\subsection{Mappings of $\mathbb{S}^1$ to Itself}

$$ \pi_1 \mathcal{S}^1 = [ (\mathcal{S}^1, \cdot), (\mathcal{S}^1, \cdot) ] $$


\section{Homotopy Types}

\attr{Lecture 7.}

\textit{Let $\top{X}{\tau}$ $\top{Y}{\sigma}$.}

A Homotopy Type is a synonym to homotopy equivalences between spaces. It is {\bf \large not} the homotopy type as in Homotopy Type Theory.

\claim{Homeomorphisms and topological spaces form a category.}
\claim{Homeomorphisms and topological spaces form a group (for all morphisms with one side fixed).}


$Homeo(X)$ group of all identity maps of X

\texttt{g} is a retraction of $Y$ to $X$ on $g(x)$ , g is also left inverse of f

$ g \circ f \homotopic{} id_x $

\defn{Homotopic Equivalence}{
  A mapping $f: X \to Y$ is a homotopic equivalence when $\exists g: Y \to X$ such that $ g \circ f \homotopic{} id_X $ and $ f \circ g \homotopic{} id_Y $
}.


\begin{theorem}
  Homotopy equivalence between spaces is an \nameref{def:equivalence}.
\end{theorem}

\begin{proof}
  \ldots
\end{proof}

\let\point\cdot

\begin{example}
  $f: X \to \point $ is a homotopic equivalence to a point $\iff$ X is contractible.

  $H_0 = id \homotopic{} H_1 = f$

\end{example}

  % \[
  %   \begin{tikzcd}[font=\large, row sep=huge, column sep=huge]
  %      X \arrow{r}{$f$} Y \arrow{r}{$g$} X
  %   \end{tikzcd}
  % \]

\claim{Holes matter! $\infty \homotopic{} \texttt{B} \homotopic{} \textit{o\_O}$,look at this graphically, not symbolically.}


\begin{example}
  Glueing of a disk to some space using quotienting.
  \textit{expanding/reducing?}
\end{example}

\claim{Homotopy Theory cares only about spaces glued from a finite number of $\mathcal{S}^n$ spaces, see Shape Theory.}


\defn{Simple Homotopy Type}{A homotopy type is simple if you use a finite number of expansions and reductions}

%foreshadowing
\claim{This is a cue to working with simplicial complexes.}

\subsection*{\ldots Interlude \ldots}

$f: X\to Y$, $f$ is a proper mapping, $\inv{f}(\text{compact})$ is a compact

\todo{Define compact}

Defining proper homotopy equivalences.

$\vr{n} \to \point$ is an improper homotopy. Compactness is a bitch.



\section{Homeomorphisms}

A fact about a group of homeomorphisms of an interval: homotopy $h_t = tf + (1-t)g$ monotonicity $\iff H^+, H^-$ are convex subset in a vector space $C(I, I)$ of all continuous functions. \textit{Prove it}.

\begin{example}
  Given curves $\gamma$ and $\delta$ that each enclose some surface.
  Is there a homeomorphism $h: \vr{2} \to \vr{2}$ s.t. $h(\gamma) = \delta$?

  \[
    \begin{tikzcd}[font=\large, row sep=huge, column sep=huge]
      \vr{2}  \arrow{rr}{\exists h?} && \vr{2} \\
      & \arrow{ul}{\gamma} I \arrow{ur}{\delta}
    \end{tikzcd}
  \]
\end{example}

\begin{example}
  Given $f(x) = x^2$. Build a homeomorphism $h: \vr{2} \to \vr{2}$ s.t. $h(y = 0) = \Gamma(f)$.
  $$h(x,y) \mapsto (x, x^2 + y)$$
  $$\inv{h}(x,y) \mapsto (x, -x^2 + y)$$
  $$\inv{h} \circ h = id_{\vr{2}}$$

  Generalizing,
  $$h(x,y) \mapsto (x, f(x) + y)$$
  $$\inv{h}(x,y) \mapsto (x, -f(x) + y)$$
  $$\inv{h} \circ h = id_{\vr{2}}$$

\end{example}

\begin{example}
  Given a graph of a continuous function $f: [a,b] \to \vr{},\, f(a) = f(b) = 0,\, m = \min f(x),\, M = \max f(x),\, \epsilon > 0$ that lies within a known rectangle $K = [a,b] \times [m - \epsilon, M + \epsilon]$.

  Prove, $\exists h: \vr{2} \to \vr{2}$ ($h$ is a homeomorphism), s.t.
  \begin{itemize}
  \item $h([a,b] \times 0) = \Gamma(f)$
  \item $h(z) = z \; \forall z \in \vr{2} \setminus K$.
  \end{itemize}

  This is how we get to locally deform curves on a plane.
\end{example}

\defn{Isotopy}{
  A homotopy $H: X \times \interval \to X$ s.t. $\forall t,\, H_t$ is a homeomorphism is an isotopy.
}

\lemma{Alexander's Theorem}{
  Any homeomorphism of a unit interval (or a disk, or any convex set) which is fixed on its boundary $\quot{h}{\partial\mathcal{D}^2} = id_{\mathcal{D}^2}$ is isotopic to $id_I$ with an isotopy that is also fixed on is boundary.

  E.g. there is a homeomorphism that maps a rendering of a letter A inside a disk to rendering of a letter R.
}{\textsc{\href{https://en.wikipedia.org/wiki/Alexander's_trick}{Alexander's Trick}}.

  $H: \mathcal{D}^2 \times [0,1] \to \mathcal{D}^2$ is an isotopy.

}

Important result: if you have two continuous deformations of a unit disk, you can have a third continuous deformations between them.
It is possible to use differentiability.

See also, Cerf's theorem, \url{https://en.wikipedia.org/wiki/Cerf_theory} and Smale's theorem (don't confuse with Poincare conjecture).

\section{Glueing and Complexes}

We were building up Homotopy Theory definitions to work with CW complexes.

\defn{CW complex}{\label{def:cw-complex}
  Also called a {\bf cell complex}. A space, created by glueing simple spaces together.
}

Quotienting is the workhorse of glueing. Use quotient topology to set up topologies after glueing two spaces.

\begin{example}
  Identifying all points from a boundary of a disk (set $X$) and a single point from set $Y$ as a single equivalence class.
\end{example}

\defn{Glueing}{\label{def:glueing}

  Notation: $ X \glue_{f} A $
}

Big idea: geometric objects have combinatorial structure, there is a lot of machinery that lets you ``dumb down'' to those structures to simplify working with objects equivalent \upto{up to} a homotopy.

Foreshadowing: \href{https://en.wikipedia.org/wiki/Whitehead_theorem}{Whitehead theorem}.

\begin{example}
Let's build a cube.
Its vertices are 8 zero-dimensional disks: $\mathcal{D}^0$, edges are 1-dimensional disks, faces are 2-dimensional disks.

(NB spaces like $K^1$ are enough to model graphs in topology)

\begin{align*}
  K^0 &= \bigsqcup \mathcal{D}_i^0\\
  K^1 &= \left(\bigsqcup \mathcal{D}_i^1\right) \glue_{\phi_i} K^0\\
  K^2 &= \left(\bigsqcup \mathcal{D}_i^2\right) \glue_{\psi_i} K^1
\end{align*}

$K^i$ is an i-th skeleton. \textit{Interiors} of disks are called {\bf cells}.

\end{example}

\claim{An integer line is a cell complex.}

\defn{CW complex}{
  \begin{itemize}
  \item C is for closure-finite (closure of each cell is contained in a union of finitely many cells)
  \item W is for weak topology (roughly, every $K^i$ has a quotient topology)
  \end{itemize}
}

\section{Homotopy Groups}

% April 3.

Notation abuse: there's actually a difference between $\circ$ (map composition) and $\star$ (group operation, which is denoted as $\circ$ instead!)


% TODO: generalize
\let\oldpi\pi
\renewcommand{\pi}{\hyperref[def:fundamental-group]{\oldpi}}

\defn{Fundamental group}{\label{def:fundamental-group}

  \url{https://en.wikipedia.org/wiki/Homotopy_group}

  $$ \pi_1(\topX, x_0) = [(\nsphere{1}, \cdot), (\topX, x_0)] = [(I, \partial I), (\topX, x_0)]$$ %

  \todo{where [A, B] are homotopy classes.}

  The group multiplication is defined as homotopy composition for {\bf spaces equivalent up to a homotopy}:
  $$ (\alpha \circ \beta)(t) = \alpha(2t) t \in [0, {1\over 2}, \beta(2t - 1) t \in [{1\over 2}, 1]] $$

  Neutral element suggestions: constant mapping doesn't work.

  Let's use deformations up to a homotopy:
  $$ [\alpha] \circ [\beta] = [\alpha \circ \beta] $$

  $$[\alpha] = [\alpha^\prime] \implies \alpha \homotopic{} \alpha^\prime$$
}

\todo{Now prove it's actually a group}

\begin{theorem}
  $$
  [\alpha] \circ [\epsilon] =  [\epsilon] \circ [\alpha]  = [\alpha]
  $$
\end{theorem}
\begin{proof}
  In other words, $\alpha \circ \epsilon \simeq \alpha$
\end{proof}

Big idea: we're handling an equivalence class by introducing an extra parameter for a class deformation, and defining operations as parametrized homotopies. Lesson: paying for definitions is hard.

\ldots

% (get-buffer-window "monkey")
% (delete-window nil)
%%


\begin{theorem}
  \todo{This has been discussed earlier}
  $$ \pi_1 (\nsphere{1}) \cong \mathbb{Z} $$
\end{theorem}


\subsection{Homotopy Groups of Spheres}

The i-th homotopy group $\pi_i(\nsphere{n})$ summarizes the different ways in which the i-dimensional sphere $\nsphere{i}$
can be mapped continuously (in homotopical sense, i.e. maps up to a homotopy) into the n-dimensional sphere $\nsphere{n}$.


\subsection{Computing the Fundamental Group}

% April 10.

Problems: given a space, compute its group. Given a group, compute its space.

%   For any finite group $G$, build a 2-dimensional CW-complex $X$, s.t. $\pi_1(X, x) = G$.
% Gave birth to combinatorial group theory (see also: word problem).


Two instruments: Cover (a special case of a fibration) and van Kampen's Theorem

\begin{theorem}
  {van Kampen}

  \url{https://en.wikipedia.org/wiki/Seifert-van_Kampen_theorem} \todo{we were discussing the combinatorial definition}

  Big idea: express the structure of the fundamental group through the fundamental groups of two open, path-connected subspaces $A$ and $B$ that cover $X$.

  Let $X$ is a path connected space, and $A, B$ are its subspaces such that $X = A \bigcup B$ and $A \bigcap B$ are
  path connected. ($\bigcup$ is generalizable by a product, $\times$)

  \[
    \pi_1 (A \times B) = \pi_1 A \times \pi_1 B
  \]

  where $\times$ is an amalgamated free product.

\end{theorem}


Categorically: the fundamental group is a terminal object in some category.


\begin{example}
  $$ \pi_1 (\nsphere{1} \wedgesum  \nsphere{1}) = \integer \times \integer = \mathcal{F}_2  $$
\end{example}

\begin{example}
  A 2-disk is path connected.

  $$ \pi_1 (\ndisk{2}) = 0 $$
\end{example}

\begin{example}
  $$ \pi_1(\nsphere{2}) = 0 $$

  $A$ and $B$ are both $\ndisk{2}$.
\end{example}

\subsubsection{Computing Fundamental Groups of Surfaces}

\begin{example}
  {\nameref{def:fundamental-group} of $\ntorus{2}$}

  This is solvable if you look at the torus as a \cwcomplex --- note there are multiple ways to compute a complex for a torus. For example, try using a diagonal of a square.

  $$ \pi_1{\ntorus{2}} = \angled{a,b \mid aba^{-1}b^{-1}} = \integer \oplus \integer $$
\end{example}

% In category theory, a branch of mathematics, a pullback (also called a fiber product, fibre product, fibered product or Cartesian square) is the limit of a diagram consisting of two morphisms f : X → Z and g : Y → Z with a common codomain. The pullback is often written






\section{Resources}

Course page: \url{https://sites.google.com/site/kafedramatematikikau/products-services/homotopy-theory}

\bibliography{references}{}
\bibliographystyle{apalike}


\end{document}
