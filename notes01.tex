\documentclass[10pt]{article}
\usepackage[a4paper, margin=1in, top=0.7in, columnsep=2cm]{geometry}
\usepackage[yyyymmdd,hhmmss]{datetime}
\usepackage{hyperref}
\usepackage{cool}
\usepackage{fancyhdr}
\usepackage{amsthm}

\usepackage[makeroom]{cancel}

\usepackage{titling}
\setlength{\droptitle}{-3em}   % This is your set screw

\hypersetup{
    colorlinks,
    linkcolor={red!30!black},
    citecolor={blue!50!black},
    urlcolor={blue!80!black}
}

\newtheoremstyle{defstyle}% <name>
{6pt}% <Space above>
{6pt}% <Space below>
{}% <Body font>
{}% <Indent amount>
{}% <Theorem head font>
{:}% <Punctuation after theorem head>
{.5em}% <Space after theorem headi>
{\thmname{\textbf{#1}}\thmnumber{ #2}\thmnote{ (\textsc{#3})}}


\theoremstyle{defstyle}
\newcounter{axiom} \numberwithin{axiom}{section}
\newtheorem{theorem}{Theorem}[section]
\newtheorem{elemma}[theorem]{Lemma}%[section]
\newtheorem{eclaim}[theorem]{Claim}
\newtheorem{eaxiom}[axiom]{Axiom}
\newtheorem{definition}[theorem]{Definition}

\newcommand\claim[1]{
  \begin{eclaim}
    #1
  \end{eclaim}
  \bigskip
}

\newcommand\nclaim[2]{
  \begin{eclaim}[#1]
    #2
  \end{eclaim}
  \bigskip
}

\newcommand\defn[2]{
  \begin{definition}[#1]
    #2
  \end{definition}
  \medskip
}

\newcommand\axiom[2]{
  \begin{eaxiom}[#1]
    #2
  \end{eaxiom}
  \medskip
}

\newcommand\lemma[3]{
  \begin{elemma}[#1]
    #2
  \end{elemma}
  \begin{proof}
    #3
  \end{proof}
  \medskip
}

\pagestyle{fancy}
\lhead{}
\chead{}
\rhead{}
\lfoot{\footnotesize {\yyyymmdddate\today} at \currenttime}
\cfoot{}
\rfoot{\thepage}
\renewcommand{\headrulewidth}{0.0pt}
\renewcommand{\footrulewidth}{0.0pt}

\usepackage{graphicx}
\usepackage{float} % use [H] for placing
\graphicspath{ {figures/} }

\usepackage{amsmath,amsthm,amssymb}
\usepackage{mathtools}
\usepackage{xcolor}
\usepackage{hyperref}
\usepackage{xspace}
\usepackage{comment}
\usepackage{url} % for bib entries
\usepackage{fancyhdr}

%% no paragraph offsets
\usepackage{parskip}
\setlength{\parskip}{0.1cm}
\setlength{\parindent}{0mm}

\newcommand{\rd}[1]{\ref{def:#1}}

\newcommand{\todo}[1]{\textcolor{gray}{\bf{#1}}} % TODO note
\newcommand{\attr}[1]{\textcolor{gray}{\bf{#1}}} % attribution

% bold vector
\renewcommand{\vec}[1]{\mathbf{#1}}

% common definitions

\newcommand*{\op}[1]{\operatorname{#1}}
\newcommand*{\vr}[1]{\mathbb{R}^{#1}}
\newcommand\N{\mathcal{N}}
\newcommand\inv[1]{{#1}^{-1}}
\newcommand\abs[1]{\left|#1\right|}
\newcommand\e[1]{\mathrm{e}^{#1}}
\renewcommand\exp[1]{\mathbf{exp}\,\left\{#1\right\}}
\newcommand\like{\propto}


\title{Introduction to Homotopy Theory \\
  \large (Notes based on lectures by Sergiy Maksymenko)
}
\author{}
\date{}


\begin{document}

\maketitle
\thispagestyle{fancy}

\begin{abstract}
  Homotopy theory studies spaces up to a homotopy, which a continuous deformation of one
  continuous function to another. This documents is a work in progress done during a course audit.
  These notes are taken purposefully in English to strenghten intuition
  and simplify lookup of concepts in related literature.
\end{abstract}

\tableofcontents

\section{Notes on Point-set Topology}

\defn{Topological space}{
  A topological space is a pair $\langle X, \tau \rangle$, where $X$ is a set
  and $\tau$, a topology on X, is a collection of subsets ($\tau \subseteq \mathcal{P}(X)$)
  called open sets, such that:

  \begin{itemize}
  \item $\emptyset \in \tau$.
  \item $X \in \tau$.
  \item $\tau$ is closed under arbitrary finite intersections.
  \item $\tau$ is closed under abbitrary unions. \todo{Maybe find some cute notation for this.}
  \end{itemize}
}

\defn{Trivial Topology}{
  A topological space is called trivial, when the topology on $X$
  consists only of $\emptyset$ and $X$.
}

\defn{Discrete Topology}{
  A topological space is called discrete, when $\tau = \mathcal{P}(X)$.
}

\defn{Continuous map}{
  Let $\langle X, \tau \rangle$ and $\langle Y, \sigma \rangle$ be topological spaces.
  A map $f: X \to Y$ is \textbf{continuous} if $\forall s \in \sigma, \inv{f}(s) \in \tau$.
  In plain English, a map is continuous when a preimage of an open set in $Y$ is an open set in $X$.
}

\todo{Base of topology and methods of inducing topologies on sets were discussed during Lecture 2.}

\todo{Bonus. Compare topology to a field of sets to a $\sigma$-algebra to a Borel $\sigma$-algebra. Discussed during Lecture 5.}


\section{Homotopy}

\subsection{Definition}
\subsection{Examples}
\subsection{Homotopy and equivalence}

\begin{theorem}
  Homotopy equivalence between spaces is an equivalence relation.
\end{theorem}

\begin{proof}
  \ldots
\end{proof}

\begin{theorem}
  A homotopy between continuous maps is an equivalence relation.
\end{theorem}

\begin{proof}
  \ldots
\end{proof}

\subsection{Contractible Spaces}

\subsubsection{Convex Sets}

\subsubsection{Star-convex Sets}

\subsubsection{Trees}

\section{Quotient Spaces and Maps}

\todo{See lectures 2 and 3.}

\todo{Brouwer's Fixed Point Theorem was mentioned around here. Bring up the context?}.


\subsection{Quotients and Groups}

\subsection{Cones of Topological Spaces}

\section{Retractions and Deformation Retractions}

\section{Classes of Homotopy Maps}

\todo{See lectures 4 and 5}.

\subsection{Mappings of $\mathbb{S}^1$ to Itself}


\section{Resources}

Course page: \url{https://sites.google.com/site/kafedramatematikikau/products-services/homotopy-theory}

\bibliography{references}{}
\bibliographystyle{apalike}



\end{document}
